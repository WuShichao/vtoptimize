\documentclass[modern]{aastex63}

\usepackage{acro}
\usepackage{amsmath}
\usepackage{amssymb}

\newcommand{\dd}{\mathrm{d}}
\newcommand{\diff}[2]{\frac{\dd #1}{\dd #2}}

\DeclareAcronym{BBH}{
  short = {BBH},
  long = {binary black hole}
}
\DeclareAcronym{BNS}{
  short = {BNS},
  long = {binary neutron star}
}
\DeclareAcronym{EM}{
  short = {EM},
  long = {electromagnetic}
}
\DeclareAcronym{GW}{
  short = {GW},
  long = {gravitational wave}
}
\DeclareAcronym{IFO}{
  short = {IFO},
  long = {interferometer}
}
\DeclareAcronym{SNR}{
  short = {SNR},
  long = {signal to noise ratio}
}

\begin{document}

\title{Strategies for Tuning Downtime and Duty Cycle to Optimize Gravitational
Wave Detections}
\author[0000-0003-1540-8562]{Will M. Farr}
\email{will.farr@stonybrook.edu}
\affiliation{Department of Physics and Astronomy, Stony Brook University, Stony Brook NY 11794, United States}
\affiliation{Center for Computational Astronomy, Flatiron Institute, 162 5th Ave, New York NY 10010, United States}

\begin{abstract} I give some order-of-magnitude arguments about how to optimize
the duty cycle of multiple, heterogeneous GW detectors, to maximize the number
of detections.  I consider networks that are representative of likely
configurations in the near future and mid-2020s. \end{abstract}

\section{Preliminaries}

It is almost always the right choice to optimize the number of detections,
$N_\mathrm{det}$ which scales as
%
\begin{equation}
  N_\mathrm{det} \propto V T,
\end{equation}
%
where $V$ is the sensitive volume for a particular source type, and $T$ is the
overall uptime.  In a Euclidean universe (appropriate for \ac{BNS} events, and
even a good approximation for \ac{BBH} events up until the 3G era),
%
\begin{equation}
  V \propto \rho^3
\end{equation}
%
where $\rho$ is the \ac{SNR} of a representative source at a fixed distance. Due
to the cubic scaling with \ac{SNR}, for a single detector, it is almost always
the right decision to trade a small amount of downtime for a sensitivity
improvement that boosts the \ac{SNR}.

For multiple detectors, however, the issue of scheduling of downtimes has an
effect.  In a multi-detector network, \acp{SNR} add in quadrature
%
\begin{equation}
  \rho = \sqrt{\sum_{i=1}^{N_\mathrm{IFO}} \rho_i^2},
\end{equation}
%
and the overall observation time depends on the scheduling strategy.

Here we will assume that the sensitive volume is dominated by the \ac{GW}
detection horizon, not by a simultaneous \ac{EM} horizon; in this latter case,
optimizing the uptime is the only consideration.

\section{Additional Less-Sensitive Detector}

Consider the scheduling problem for the addition of one more detector to an
existing network of $N$ \acp{IFO}, where the additional detector is a factor $f
\ll 1$ less sensitive than the typical detector in the network and operates with
a duty cycle $T_\mathrm{obs} = (1-\tau) T_\mathrm{wall}$ (e.g.\ the current VIGO
detector has $f \simeq 1/3$).  If we schedule the detector downtime
synchronously with the remainder of the network (here assumed to have comparable
duty cycle), then the total $VT$ is
%
\begin{equation}
  \label{eq:low-sens-synch}
  VT \propto (1-\tau) \left( 1 + f^2 \right)^{3/2} \simeq (1-\tau) \left( 1 + \frac{3}{2} f^2 \right).
\end{equation}
%
On the other hand, if the downtime of this additional detector is scheduled
randomly, then we have
%
\begin{equation}
  VT \propto (1-\tau)^N \tau + N (1-\tau)^{N} \tau \left( 1 - \frac{1}{N} + f^2 \right)^{3/2} + (1-\tau)^{N+1} \left( 1 + f^2 \right)^{3/2} + \mathcal{O}\left(\tau^2\right).
\end{equation}
%
(The three cases correspond to the $N$ original \acp{IFO} operating, new
\ac{IFO} down; one of the $N$ original \acp{IFO} down, new \ac{IFO} operating;
and all \acp{IFO} operating.)  This is approximately
%
\begin{equation}
  \label{eq:low-sens-random}
  VT \propto \left( 1 - \left[ \frac{3}{2} + \frac{3}{8N} \right] \tau \right) \left( 1 + \frac{3}{2} f^2\right) + \mathcal{O}\left( \tau f^2 \right)
\end{equation}
%
Comparing Eq.\ \eqref{eq:low-sens-synch} and Eq.\ \eqref{eq:low-sens-random}, we
see that the synchronous strategy dominates the random-downtime strategy in all
cases.

\section{Additional Comparable-Sensitivity Detector}

Suppose we are considering whether to add an additional detector to the network
that is comparable sensitivity to the existing network of $N$.  Then in the
synchronous case we have
%
\begin{equation}
  VT \propto (1-\tau) \left(1 + \frac{1}{N}\right)^{3/2} \simeq (1-\tau) \left( 1 + \frac{3}{2N} \right).
\end{equation}
%
In the random scheduling case we have
%
\begin{multline}
  VT \propto (1-\tau)^N \tau + N (1-\tau)^N \tau + (1-\tau)^{N+1} \left(1 + \frac{1}{N}\right)^{3/2} \\ \simeq \left( 1 - \frac{3}{2} \tau \right) \left( 1 + \frac{3}{2N} \right) + \mathcal{O}\left( \frac{\tau}{N} \right)
\end{multline}
%
Once again, we see that the random-scheduling case is dominated by synchronous
scheduling case.

\section{Conclusion}

If possible, detector downtimes should be scheduled synchronously.

\bibliography{vtoptimize}

\end{document}
